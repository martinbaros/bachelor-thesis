\documentclass[12pt,a4paper,twoside,openright]{report}
\let\openright=\cleardoublepage



%%% Choose a language %%%

\newif\ifEN
\ENtrue   % uncomment this for english
%\ENfalse   % uncomment this for czech

%%% Configuration of the title page %%%

\def\ThesisTitleStyle{mff} % MFF style
%\def\ThesisTitleStyle{cuni} % uncomment for old-style with cuni.cz logo
%\def\ThesisTitleStyle{natur} % uncomment for nature faculty logo

\def\UKFaculty{Faculty of Mathematics and Physics}
%\def\UKFaculty{Faculty of Science}

\def\UKName{Charles University in Prague} % this is not used in the "mff" style

% Thesis type names, as used in several places in the title
\def\ThesisTypeTitle{\ifEN BACHELOR THESIS \else BAKALÁŘSKÁ PRÁCE \fi}
%\def\ThesisTypeTitle{\ifEN MASTER THESIS \else DIPLOMOVÁ PRÁCE \fi}
%\def\ThesisTypeTitle{\ifEN RIGOROUS THESIS \else RIGORÓZNÍ PRÁCE \fi}
%\def\ThesisTypeTitle{\ifEN DOCTORAL THESIS \else DISERTAČNÍ PRÁCE \fi}
\def\ThesisGenitive{\ifEN bachelor \else bakalářské \fi}
%\def\ThesisGenitive{\ifEN master \else diplomové \fi}
%\def\ThesisGenitive{\ifEN rigorous \else rigorózní \fi}
%\def\ThesisGenitive{\ifEN doctoral \else disertační \fi}
\def\ThesisAccusative{\ifEN bachelor \else bakalářskou \fi}
%\def\ThesisAccusative{\ifEN master \else diplomovou \fi}
%\def\ThesisAccusative{\ifEN rigorous \else rigorózní \fi}
%\def\ThesisAccusative{\ifEN doctoral \else disertační \fi}



%%% Fill in your details %%%

% (Note: \xxx is a "ToDo label", which makes the unfilled visible. Remove it.)
\def\ThesisTitle{BrickSnoop: Optimizer of LEGO{\textregistered} Brick Orders}
\def\ThesisAuthor{Martin Baroš}
\def\YearSubmitted{2023}

% department assigned to the thesis
\def\Department{Department of Software Engineering}
% Is it a department (katedra), or an institute (ústav)?
\def\DeptType{Department}

\def\Supervisor{Mgr. Ing. Robert Husák}
\def\SupervisorsDepartment{Department of Software Engineering}

% Study programme and specialization
\def\StudyProgramme{Computer Science}
\def\StudyBranch{Programming and software development (IPP2)}

\def\Dedication{%
Dedication. \xxx{It is nice to say thanks to supervisors, friends, family, book authors and food providers.}
}

\def\AbstractEN{%
\xxx{Abstracts are an abstract form of art. Use the most precise, shortest sentences that state what problem the thesis addresses, how it is approached, pinpoint the exact result achieved, and describe the applications and significance of the results. Highlight anything novel that was discovered or improved by the thesis. Maximum length is 200 words, but try to fit into 120. Abstracts are often used for deciding if a reviewer will be suitable for the thesis; a well-written abstract thus increases the probability of getting a reviewer who will like the thesis.}
% ABSTRACT IS NOT A COPY OF YOUR THESIS ASSIGNMENT!
}

\def\AbstractCS{%
\xxx{You will need to submit both Czech and English abstract to the SIS, no matter what language you use in the thesis. If writing in English, translate the contents of \texttt{\textbackslash{}AbstractEN} into this field. In case you do not speak czech, your supervisor should be able to help you with the translation.}
}

% 3 to 5 keywords (recommended), each enclosed in curly braces.
% Keywords are useful for indexing and searching for the theses by topic.
\def\Keywords{%
\xxx{{key} {words}}
}

% If your abstracts are long and do not fit in the infopage, you can make the
% fonts a bit smaller by this setting. (Also, you should try to compress your abstract more.)
% Alternatively, consider increasing the size of the page by uncommenting the
% geometry modification in thesis.tex.
\def\InfoPageFont{}
%\def\InfoPageFont{\small}  %uncomment to decrease font size

\ifEN\relax\else
% If you are writing a czech thesis, you additionally need to fill in the
% english translation of the metadata here!
\def\ThesisTitleEN{\xxx{Thesis title in English}}
\def\DepartmentEN{\xxx{Name of the department in English}}
\def\DeptTypeEN{\xxx{Department}}
\def\SupervisorsDepartmentEN{\xxx{Superdepartment}}
\def\StudyProgrammeEN{\xxx{study programme}}
\def\StudyBranchEN{\xxx{study branch}}
\def\KeywordsEN{%
\xxx{{key} {words}}
}
\fi


\usepackage[a-2u]{pdfx}

\ifEN\else\usepackage[czech,shorthands=off]{babel}\fi
\usepackage[utf8]{inputenc}
\usepackage[T1]{fontenc}

% See https://en.wikipedia.org/wiki/Canons_of_page_construction before
% modifying the size of printable area. LaTeX defaults are great.
% If you feel it would help anything, you can enlarge the printable area a bit:
%\usepackage[textwidth=390pt,textheight=630pt]{geometry}
% The official recommendation expands the area quite a bit (looks pretty harsh):
%\usepackage[textwidth=145mm,textheight=247mm]{geometry}

%%% FONTS %%%
\usepackage{lmodern} % TeX "original" (this sets up the latin mono)

% Optionally choose an override for the main font for typesetting:
\usepackage[mono=false]{libertinus} % popular for comp-sci (ACM uses this)
%\usepackage{tgschola} % Schoolbook-like (gives a bit of historic feel)
%\usepackage[scale=0.96]{tgpagella} % Palladio-like (popular in formal logic).
% IBM Plex font suite is nice but requires us to fine-tune the sizes, also note
% that it does not directly support small caps (\textsc) and requires lualatex:
%\usepackage[usefilenames,RM={Scale=0.88},SS={Scale=0.88},SScon={Scale=0.88},TT={Scale=0.88},DefaultFeatures={Ligatures=Common}]{plex-otf}

% Optionally choose a custom sans-serif fonts (e.g. for figures and tables).
% Default sans-serif font is usually Latin Modern Sans. Some font packages
% (e.g. libertinus) replace that with a better matching sans-serif font.
%\usepackage{tgheros} % recommended and very readable (Helvetica-like)
%\usepackage{FiraSans} % looks great
% DO NOT typeset the main text in sans-serif font!
% The serifs make the text easily readable on the paper.

% IMPORTANT FONT NOTE: Some fonts require additional PDF/A conversion using
% the pdfa.sh script. These currently include only 'tgpagella'; but various
% other fonts from the texlive distribution need that too (mainly the Droid
% font family).


% some useful packages
\usepackage{microtype}
\usepackage{amsmath,amsfonts,amsthm,bm}
\usepackage{graphicx}
\usepackage{xcolor}
\usepackage{booktabs}
\usepackage{caption}
\usepackage{floatrow}

% load bibliography tools
\usepackage[backend=bibtex,natbib,style=numeric,sorting=none]{biblatex}
% alternative with alphanumeric citations (more informative than numbers):
%\usepackage[backend=bibtex,natbib,style=alphabetic]{biblatex}
%
% alternatives that conform to iso690
% (iso690 is not formally required on MFF, but may help elsewhere):
%\usepackage[backend=bibtex,natbib,style=iso-numeric,sorting=none]{biblatex}
%\usepackage[backend=bibtex,natbib,style=iso-alphabetic]{biblatex}
%
% additional option choices:
%  - add `giveninits=true` to typeset "E. A. Poe" instead of full Edgar Allan
%  - `terseinits=true` additionaly shortens it to nature-like "Poe EA"
%  - add `maxnames=10` to limit (or loosen) the maximum number of authors in
%    bibliography entry before shortening to `et al.` (useful when referring to
%    book collections that may have hundreds of authors)
%  - for additional flexibility (e.g. multiple reference sections, etc.),
%    remove `backend=bibtex` and compile with `biber` instead of `bibtex` (see
%    Makefile)
%  - `sorting=none` causes the bibliography list to be ordered by the order of
%    citation as they appear in the text, which is usually the desired behavior
%    with numeric citations. Additionally you can use a style like
%    `numeric-comp` that compresses the long lists of citations such as
%    [1,2,3,4,5,6,7,8] to simpler [1--8]. This is especially useful if you plan
%    to add tremendous amounts of citations, as usual in life sciences and
%    bioinformatics.
%  - if you don't like the "In:" appearing in the bibliography, use the
%    extended style (`ext-numeric` or `ext-alphabetic`), and add option
%    `articlein=false`.
%
% possibly reverse the names of the authors with the default styles:
%\DeclareNameAlias{default}{family-given}

% load the file with bibliography entries
\addbibresource{refs}

% remove this if you won't use fancy verbatim environments
\usepackage{fancyvrb}

% remove this if you won't typeset TikZ graphics
\usepackage{tikz}
\usetikzlibrary{positioning} %add libraries as needed (shapes, decorations, ...)

% remove this if you won't typeset any pseudocode
\usepackage{algpseudocode}
\usepackage{algorithm}

% remove this if you won't list any source code
\usepackage{listings}


\hypersetup{unicode}
\hypersetup{breaklinks=true}

\usepackage[noabbrev]{cleveref}

\input{todos} % remove this before compiling the final version

\input{macros} % use this file for various custom definitions


\begin{document}

\include{title}

\tableofcontents

\include{intro}
\chapter{Introduction}
\label{chap:analysis}

\xxx{uviest ho do toho co sa bude diat v intro, zaviest slovnik pojmov}

\xxx{
    popis do podrobna az v diskusii
}

\section{Current workflow}
\label{sec:workflow}

This section will present the current process of creating and building LEGO creations. The usual workflow consists of creating or finding a model, finding the needed parts and buying them.

\subsection{Obtaining a model}

For the purpose of model creation, there are only two currently used tools \todo{z kade to viem musim povedat? zmiernit tvdenie an to ze toto su}:
\begin{itemize}
    \item BrickLink Studio \todo{Do I need to link this? ano popisat ze ktore to su a ktore podporuje ktore} - downloadable program
    \item MecaBrick - an online tool
\end{itemize}

\noindent After finishing modelling, we can export parts into one of the usual data formats \todo{je toto ok? kam ich vypisat?} and the chosen tool also generates a building manual for the creation.

 However, LEGO's big community of enthusiasts has already created thousands of models. Their creations are mainly published on these websites:

\begin{itemize}
    \item Rebrickable - the most popular one, also providing the opportunity for the creators to sell their model to others
    \item BrickLink Studio Gallery - some models do not provide a parts list
\end{itemize}

\noindent Chosen model usually includes a parts list exportable into a usual data format and building instructions.

\subsection{Finding the needed parts}

Having found the desired model, there are currently four options from where we can buy them.

\subsubsection{BrickLink}
\label{BrickLink}

BrickLink is currently the leading marketplace for LEGO bricks \cite{BrickLinkAbout}. Besides its most common usage of buying parts, it also provides the functionality of Wanted Lists \todo{Pridat do vocab?}, which are used to keep track of the bricks you need. Furthermore, there is a collector's sets marker, and it owns BrickLink Studio and BrickLink Studio Gallery \todo{mam to pripomenut ak to je literally pol strany vyssie?}. Its coverage of these areas makes it an easy recommendation and a go-to place for most enthusiasts.

\subsubsection{BrickOwl}
\label{BrickOwl}

BrickOwl is a marketplace with the support of Wanted Lists. Its lack of other features of BrickLink makes it less favoured, yet many stores only offer their bricks on BrickOwl \cite{BrickOwlStores}, so checking the price on both of these websites has been a part of the usual workflow for the past years.

\subsubsection{Pick a Brick}
\label{Pick a Brick}

Pick a Brick is an official LEGO service providing an opportunity to get the needed part. However, the offer of bricks highly depends on LEGO's current sets. Also, it was unavailable for customers from smaller countries such as Slovakia until recently.

\subsubsection{Independent stores}

Independent stores provide another option for finding the needed parts, but their limited offerings and difficulty finding them make this a very unpopular choice.

\subsection{Buying the needed parts}
After finding all the necessary parts, we need to buy them; however, the same brick can be bought from hundreds of stores on BrickLink and BrickOwl and can also be found on Pick a Brick. Here comes the problem of choosing the right vendor. As the number of different bricks in our creation can vary from tens to hundreds, this can take very long. Also, the stores can be located worldwide and have different shipping prices.

Fortunately, both BrickLink and BrickOwl offer the Auto Buy feature, which, using some algorithm, selects a subset of stores. Alternatively, we can use Rebrickable's cross-platform Auto Buy, which combines BrickLink and BrickOwl store offerings. However, much work will remain to be done as it does not consider the minimum price set by every store, so we have to check every store and adjust the selection accordingly. Also, there is no such tool for Pick a Brick and finding a price for a wanted part remains a non-automatic process.



\section{Issues with the current workflow}
\label{sec:problemsWorkflow}

\todo{ako povedat ze toto mi pisali ludia?}

As user research among LEGO enthusiasts suggests, there are several issues with the Auto Buy feature, mainly:

\begin{itemize}
    \item store selection is limited to the website used
    \item it does not take minimum buy into consideration
    \item it does not take shipping into consideration
    \item there is no Pick a Brick integration
    \item user unfriendliness - both function and user interface-wise
\end{itemize}

\noindent These problems tend to make people painstakingly compare all the stores manually or overpay by buying all the bricks from one big vendor while saving time.


\section{Proposed solution}
\label{sec:proposedSolution}

Creating a user-friendly web app for cross-platform LEGO brick shopping focusing on usability.

\xxx{mozno aj nieco viac by sa tu dalo povedat ale neviem ci user requirements nedat skor neskor v inej kapitole}


\include{ch1}
\include{ch2}
\include{ch3}
\include{conclusion}
\include{bibliography}

\appendix
\include{howto}

% if your attachments are complicated, describe them in a separate appendix
%\include{attachments}

\openright
\end{document}
