\chapter{Introduction}
\label{chap:analysis}

\xxx{uviest ho do toho co sa bude diat v intro, zaviest slovnik pojmov}

\xxx{
    popis do podrobna az v diskusii
}

\section{Current workflow}
\label{sec:workflow}

This section will present the current process of creating and building LEGO creations. The usual workflow consists of creating or finding a model, finding the needed parts and buying them.

\subsection{Obtaining a model}

For the purpose of model creation, there are only two currently used tools \todo{z kade to viem musim povedat? zmiernit tvdenie an to ze toto su}:
\begin{itemize}
    \item BrickLink Studio \todo{Do I need to link this? ano popisat ze ktore to su a ktore podporuje ktore} - downloadable program
    \item MecaBrick - an online tool
\end{itemize}

\noindent After finishing modelling, we can export parts into one of the usual data formats \todo{je toto ok? kam ich vypisat?} and the chosen tool also generates a building manual for the creation.

 However, LEGO's big community of enthusiasts has already created thousands of models. Their creations are mainly published on these websites:

\begin{itemize}
    \item Rebrickable - the most popular one, also providing the opportunity for the creators to sell their model to others
    \item BrickLink Studio Gallery - some models do not provide a parts list
\end{itemize}

\noindent Chosen model usually includes a parts list exportable into a usual data format and building instructions.

\subsection{Finding the needed parts}

Having found the desired model, there are currently four options from where we can buy them.

\subsubsection{BrickLink}
\label{BrickLink}

BrickLink is currently the leading marketplace for LEGO bricks \cite{BrickLinkAbout}. Besides its most common usage of buying parts, it also provides the functionality of Wanted Lists \todo{Pridat do vocab?}, which are used to keep track of the bricks you need. Furthermore, there is a collector's sets marker, and it owns BrickLink Studio and BrickLink Studio Gallery \todo{mam to pripomenut ak to je literally pol strany vyssie?}. Its coverage of these areas makes it an easy recommendation and a go-to place for most enthusiasts.

\subsubsection{BrickOwl}
\label{BrickOwl}

BrickOwl is a marketplace with the support of Wanted Lists. Its lack of other features of BrickLink makes it less favoured, yet many stores only offer their bricks on BrickOwl \cite{BrickOwlStores}, so checking the price on both of these websites has been a part of the usual workflow for the past years.

\subsubsection{Pick a Brick}
\label{Pick a Brick}

Pick a Brick is an official LEGO service providing an opportunity to get the needed part. However, the offer of bricks highly depends on LEGO's current sets. Also, it was unavailable for customers from smaller countries such as Slovakia until recently.

\subsubsection{Independent stores}

Independent stores provide another option for finding the needed parts, but their limited offerings and difficulty finding them make this a very unpopular choice.

\subsection{Buying the needed parts}
After finding all the necessary parts, we need to buy them; however, the same brick can be bought from hundreds of stores on BrickLink and BrickOwl and can also be found on Pick a Brick. Here comes the problem of choosing the right vendor. As the number of different bricks in our creation can vary from tens to hundreds, this can take very long. Also, the stores can be located worldwide and have different shipping prices.

Fortunately, both BrickLink and BrickOwl offer the Auto Buy feature, which, using some algorithm, selects a subset of stores. Alternatively, we can use Rebrickable's cross-platform Auto Buy, which combines BrickLink and BrickOwl store offerings. However, much work will remain to be done as it does not consider the minimum price set by every store, so we have to check every store and adjust the selection accordingly. Also, there is no such tool for Pick a Brick and finding a price for a wanted part remains a non-automatic process.



\section{Issues with the current workflow}
\label{sec:problemsWorkflow}

\todo{ako povedat ze toto mi pisali ludia?}

As user research among LEGO enthusiasts suggests, there are several issues with the Auto Buy feature, mainly:

\begin{itemize}
    \item store selection is limited to the website used
    \item it does not take minimum buy into consideration
    \item it does not take shipping into consideration
    \item there is no Pick a Brick integration
    \item user unfriendliness - both function and user interface-wise
\end{itemize}

\noindent These problems tend to make people painstakingly compare all the stores manually or overpay by buying all the bricks from one big vendor while saving time.


\section{Proposed solution}
\label{sec:proposedSolution}

Creating a user-friendly web app for cross-platform LEGO brick shopping focusing on usability.

\xxx{mozno aj nieco viac by sa tu dalo povedat ale neviem ci user requirements nedat skor neskor v inej kapitole}

